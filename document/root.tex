\documentclass[11pt,a4paper]{article}
\usepackage{isabelle,isabellesym}

% further packages required for unusual symbols (see also
% isabellesym.sty), use only when needed

\usepackage{amssymb}
  %for \<leadsto>, \<box>, \<diamond>, \<sqsupset>, \<mho>, \<Join>,
  %\<lhd>, \<lesssim>, \<greatersim>, \<lessapprox>, \<greaterapprox>,
  %\<triangleq>, \<yen>, \<lozenge>

%\usepackage{eurosym}
  %for \<euro>

%\usepackage[only,bigsqcap]{stmaryrd}
  %for \<Sqinter>

%\usepackage{eufrak}
  %for \<AA> ... \<ZZ>, \<aa> ... \<zz> (also included in amssymb)

%\usepackage{textcomp}
  %for \<onequarter>, \<onehalf>, \<threequarters>, \<degree>, \<cent>,
  %\<currency>

% this should be the last package used
\usepackage{pdfsetup}

% urls in roman style, theory text in math-similar italics
\urlstyle{rm}
\isabellestyle{it}

% for uniform font size
%\renewcommand{\isastyle}{\isastyleminor}


\begin{document}

\title{Pratt's Primality Certificates \\ (Isabelle Formalisation)}
\author{By Simon Wimmer and Lars Noschinski}
\maketitle

\begin{abstract}
In 1975 Vaughan Pratt introduced a proof system for certifying primes in his work
  "Every Prime has a Succinct Certificate"\cite{pratt1975certificate}.
  He showed that a number $p$ is prime iff a primality certificate for $p$ exists.
  By showing a logarithmic upper bound on the length of the certicates in size of the prime number,
  he concluded that the decision problem for prime numbers is in NP, in short PRIMES $\in$ NP.
  This work formalizes Pratt's proof system and shows the results mentioned before.  
  These proofs are based on Lehmer's theorem, a converse of Fermat's little theorem,
  which is proved in the first part of this work.
  As a side product of this proof, this work formalizes some algebraic properties, including:
  \begin {itemize}
  \item Some properties of the Euler phi function
  \item The notion of the order of an element
  \item The cyclicity of the multiplicative group of any finite field
  \end {itemize}
\end{abstract}

\tableofcontents

% sane default for proof documents
\parindent 0pt\parskip 0.5ex

% generated text of all theories
\input{session}

\nocite{*}

% optional bibliography
\bibliographystyle{abbrv}
\bibliography{root}

\end{document}

%%% Local Variables:
%%% mode: latex
%%% TeX-master: t
%%% End\dots
